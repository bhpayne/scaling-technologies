\chapter{Introduction\label{sec:introduction}}

Context: Each of these entries below is a plateau. 
Changes up or down occur due to 
\begin{itemize}
    \item skillset of staff
    \begin{itemize}
        \item Staffing can change if there's a business case justifying the cost/benefit.
        \item Staff can improve their own skills
    \end{itemize}
    \item amount of data, 
    \item complexity of queries, 
    \item number of queries, 
    \item skills of users, 
    \item latency
\end{itemize}

Three options at any given plateau 
\begin{itemize}
    \item Growth feedback loop - lower latency, more data, more queries 
    \item Sustainment – good enough. Staffing is barely sufficient
    \item Devolving – loss of capacity and capabilities. Loss of skilled Technical staff. Architects and implementers left long ago, maintenance staff wasn't replaced or there was insufficient training overlap
\end{itemize}


How to evaluate need for transition: 
\begin{itemize}
    \item profiling resource utilization (software and hardware), 
    \item Evaluate business use case, 
    \item forecasting customer behavior
\end{itemize}
Transitions can be premature or late. Both cause harm to the business.


The book is necessarily sequential, but the underlying data structure is a graph walking through a multidimensional parameter space: 
\begin{itemize}
    \item size of data, 
    \item latency per task, 
    \item compute per task, 
    \item number of queries 
\end{itemize}

